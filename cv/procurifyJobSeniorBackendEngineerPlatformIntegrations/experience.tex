\cvsection{Experience}
\begin{cventries}

\cventry%
	{Data Engineer --- Platform \& Integration Focus}
	{Ontario Public Service --- Ministry of the Attorney General}
	{Toronto, Canada}
	{2024 --- Present}
	{
		\begin{cvitems}
			\item Architected and built integration platform connecting legacy mainframe systems with modern cloud infrastructure, saving \$300K+ annually through automated data synchronization and API-driven workflows replacing manual processes.
			\item Designed scalable backend services using Python to handle complex data transformations and multi-system integrations, supporting 150+ distributed locations with real-time data flow requirements.
			\item Built robust data pipeline infrastructure on Azure cloud (Databricks, Data Factory) with focus on resilient system-to-system communication, error handling, and observability patterns applicable to AWS environments.
			\item Led cross-functional engineering collaboration during \$160M+ platform migration, implementing agile methodologies and ensuring seamless integration between legacy and modern systems.
			\item Established engineering best practices including Git workflows, code review processes, and automated testing standards, improving team velocity and code quality.
			\item Mentored team members on API design patterns, distributed system architecture, and integration best practices, fostering engineering excellence culture.
		\end{cvitems}
	}

\cventry%
	{Senior Backend Engineer/Team Lead}
	{Ontario Public Service --- Treasury Board Secretariat}
	{Toronto, Canada}
	{2019 --- 2024}
	{
		\begin{cvitems}
			\item Designed and developed production Flask APIs serving as integration layer for internal data platform, enabling seamless data flow between multiple enterprise systems and supporting critical business operations.
			\item Led engineering team of 5 in building scalable backend services, implementing RESTful API standards, authentication/authorization patterns, and robust error handling for mission-critical integrations.
			\item Built complex entity resolution service using Python and NLTK, creating APIs that matched and synchronized data across disparate systems with high accuracy, reducing manual reconciliation by 90\%.
			\item Architected cloud-native backend infrastructure on Azure, implementing microservices patterns, container orchestration, and CI/CD pipelines directly transferable to AWS ecosystem.
			\item Developed internal platform tools with Vue.js frontend and Python/Node.js backend APIs, demonstrating full-stack capabilities while maintaining backend engineering focus.
			\item Established API documentation standards, versioning strategies, and integration testing frameworks, ensuring maintainable and scalable platform architecture.
		\end{cvitems}
	}

\cventry%
	{Backend Developer --- Automation \& Integration}
	{Ontario Public Service --- Ministry of Government \& Consumer Services}
	{Toronto, Canada}
	{2017 --- 2019}
	{
		\begin{cvitems}
			\item Built Python-based integration service processing 7,000+ documents across multiple systems, automating complex workflows and ensuring data consistency for province-wide initiative affecting 10,000 employees.
			\item Developed backend automation tools integrating with multiple data sources (SQL databases, Excel, Word documents), creating unified API layer for downstream consumers and reducing manual processing by 70+ hours.
			\item Implemented robust data validation and error handling mechanisms in integration pipelines, ensuring reliability and data integrity across interconnected systems.
		\end{cvitems}
	}

\cventry%
	{Systems Administrator --- Infrastructure \& DevOps}
	{Soho VFX}
	{Toronto, Canada}
	{2016 --- 2017}
	{
		\begin{cvitems}
			\item Managed Linux server infrastructure and automated deployment processes using shell scripting, improving system reliability and reducing deployment time for production services.
			\item Built automation tools for system optimization and monitoring, demonstrating strong foundation in infrastructure management applicable to modern cloud environments.
		\end{cvitems}
	}

\end{cventries}